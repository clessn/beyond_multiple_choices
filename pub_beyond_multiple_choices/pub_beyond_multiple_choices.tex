% Options for packages loaded elsewhere
\PassOptionsToPackage{unicode}{hyperref}
\PassOptionsToPackage{hyphens}{url}
\PassOptionsToPackage{dvipsnames,svgnames,x11names}{xcolor}
%
\documentclass[
  authoryear,
  preprint,
  3p,
  twocolumn]{elsarticle}

\usepackage{amsmath,amssymb}
\usepackage{iftex}
\ifPDFTeX
  \usepackage[T1]{fontenc}
  \usepackage[utf8]{inputenc}
  \usepackage{textcomp} % provide euro and other symbols
\else % if luatex or xetex
  \usepackage{unicode-math}
  \defaultfontfeatures{Scale=MatchLowercase}
  \defaultfontfeatures[\rmfamily]{Ligatures=TeX,Scale=1}
\fi
\usepackage{lmodern}
\ifPDFTeX\else  
    % xetex/luatex font selection
\fi
% Use upquote if available, for straight quotes in verbatim environments
\IfFileExists{upquote.sty}{\usepackage{upquote}}{}
\IfFileExists{microtype.sty}{% use microtype if available
  \usepackage[]{microtype}
  \UseMicrotypeSet[protrusion]{basicmath} % disable protrusion for tt fonts
}{}
\makeatletter
\@ifundefined{KOMAClassName}{% if non-KOMA class
  \IfFileExists{parskip.sty}{%
    \usepackage{parskip}
  }{% else
    \setlength{\parindent}{0pt}
    \setlength{\parskip}{6pt plus 2pt minus 1pt}}
}{% if KOMA class
  \KOMAoptions{parskip=half}}
\makeatother
\usepackage{xcolor}
\setlength{\emergencystretch}{3em} % prevent overfull lines
\setcounter{secnumdepth}{5}
% Make \paragraph and \subparagraph free-standing
\ifx\paragraph\undefined\else
  \let\oldparagraph\paragraph
  \renewcommand{\paragraph}[1]{\oldparagraph{#1}\mbox{}}
\fi
\ifx\subparagraph\undefined\else
  \let\oldsubparagraph\subparagraph
  \renewcommand{\subparagraph}[1]{\oldsubparagraph{#1}\mbox{}}
\fi

\usepackage{color}
\usepackage{fancyvrb}
\newcommand{\VerbBar}{|}
\newcommand{\VERB}{\Verb[commandchars=\\\{\}]}
\DefineVerbatimEnvironment{Highlighting}{Verbatim}{commandchars=\\\{\}}
% Add ',fontsize=\small' for more characters per line
\usepackage{framed}
\definecolor{shadecolor}{RGB}{241,243,245}
\newenvironment{Shaded}{\begin{snugshade}}{\end{snugshade}}
\newcommand{\AlertTok}[1]{\textcolor[rgb]{0.68,0.00,0.00}{#1}}
\newcommand{\AnnotationTok}[1]{\textcolor[rgb]{0.37,0.37,0.37}{#1}}
\newcommand{\AttributeTok}[1]{\textcolor[rgb]{0.40,0.45,0.13}{#1}}
\newcommand{\BaseNTok}[1]{\textcolor[rgb]{0.68,0.00,0.00}{#1}}
\newcommand{\BuiltInTok}[1]{\textcolor[rgb]{0.00,0.23,0.31}{#1}}
\newcommand{\CharTok}[1]{\textcolor[rgb]{0.13,0.47,0.30}{#1}}
\newcommand{\CommentTok}[1]{\textcolor[rgb]{0.37,0.37,0.37}{#1}}
\newcommand{\CommentVarTok}[1]{\textcolor[rgb]{0.37,0.37,0.37}{\textit{#1}}}
\newcommand{\ConstantTok}[1]{\textcolor[rgb]{0.56,0.35,0.01}{#1}}
\newcommand{\ControlFlowTok}[1]{\textcolor[rgb]{0.00,0.23,0.31}{#1}}
\newcommand{\DataTypeTok}[1]{\textcolor[rgb]{0.68,0.00,0.00}{#1}}
\newcommand{\DecValTok}[1]{\textcolor[rgb]{0.68,0.00,0.00}{#1}}
\newcommand{\DocumentationTok}[1]{\textcolor[rgb]{0.37,0.37,0.37}{\textit{#1}}}
\newcommand{\ErrorTok}[1]{\textcolor[rgb]{0.68,0.00,0.00}{#1}}
\newcommand{\ExtensionTok}[1]{\textcolor[rgb]{0.00,0.23,0.31}{#1}}
\newcommand{\FloatTok}[1]{\textcolor[rgb]{0.68,0.00,0.00}{#1}}
\newcommand{\FunctionTok}[1]{\textcolor[rgb]{0.28,0.35,0.67}{#1}}
\newcommand{\ImportTok}[1]{\textcolor[rgb]{0.00,0.46,0.62}{#1}}
\newcommand{\InformationTok}[1]{\textcolor[rgb]{0.37,0.37,0.37}{#1}}
\newcommand{\KeywordTok}[1]{\textcolor[rgb]{0.00,0.23,0.31}{#1}}
\newcommand{\NormalTok}[1]{\textcolor[rgb]{0.00,0.23,0.31}{#1}}
\newcommand{\OperatorTok}[1]{\textcolor[rgb]{0.37,0.37,0.37}{#1}}
\newcommand{\OtherTok}[1]{\textcolor[rgb]{0.00,0.23,0.31}{#1}}
\newcommand{\PreprocessorTok}[1]{\textcolor[rgb]{0.68,0.00,0.00}{#1}}
\newcommand{\RegionMarkerTok}[1]{\textcolor[rgb]{0.00,0.23,0.31}{#1}}
\newcommand{\SpecialCharTok}[1]{\textcolor[rgb]{0.37,0.37,0.37}{#1}}
\newcommand{\SpecialStringTok}[1]{\textcolor[rgb]{0.13,0.47,0.30}{#1}}
\newcommand{\StringTok}[1]{\textcolor[rgb]{0.13,0.47,0.30}{#1}}
\newcommand{\VariableTok}[1]{\textcolor[rgb]{0.07,0.07,0.07}{#1}}
\newcommand{\VerbatimStringTok}[1]{\textcolor[rgb]{0.13,0.47,0.30}{#1}}
\newcommand{\WarningTok}[1]{\textcolor[rgb]{0.37,0.37,0.37}{\textit{#1}}}

\providecommand{\tightlist}{%
  \setlength{\itemsep}{0pt}\setlength{\parskip}{0pt}}\usepackage{longtable,booktabs,array}
\usepackage{calc} % for calculating minipage widths
% Correct order of tables after \paragraph or \subparagraph
\usepackage{etoolbox}
\makeatletter
\patchcmd\longtable{\par}{\if@noskipsec\mbox{}\fi\par}{}{}
\makeatother
% Allow footnotes in longtable head/foot
\IfFileExists{footnotehyper.sty}{\usepackage{footnotehyper}}{\usepackage{footnote}}
\makesavenoteenv{longtable}
\usepackage{graphicx}
\makeatletter
\def\maxwidth{\ifdim\Gin@nat@width>\linewidth\linewidth\else\Gin@nat@width\fi}
\def\maxheight{\ifdim\Gin@nat@height>\textheight\textheight\else\Gin@nat@height\fi}
\makeatother
% Scale images if necessary, so that they will not overflow the page
% margins by default, and it is still possible to overwrite the defaults
% using explicit options in \includegraphics[width, height, ...]{}
\setkeys{Gin}{width=\maxwidth,height=\maxheight,keepaspectratio}
% Set default figure placement to htbp
\makeatletter
\def\fps@figure{htbp}
\makeatother

\makeatletter
\@ifpackageloaded{caption}{}{\usepackage{caption}}
\AtBeginDocument{%
\ifdefined\contentsname
  \renewcommand*\contentsname{Table of contents}
\else
  \newcommand\contentsname{Table of contents}
\fi
\ifdefined\listfigurename
  \renewcommand*\listfigurename{List of Figures}
\else
  \newcommand\listfigurename{List of Figures}
\fi
\ifdefined\listtablename
  \renewcommand*\listtablename{List of Tables}
\else
  \newcommand\listtablename{List of Tables}
\fi
\ifdefined\figurename
  \renewcommand*\figurename{Figure}
\else
  \newcommand\figurename{Figure}
\fi
\ifdefined\tablename
  \renewcommand*\tablename{Table}
\else
  \newcommand\tablename{Table}
\fi
}
\@ifpackageloaded{float}{}{\usepackage{float}}
\floatstyle{ruled}
\@ifundefined{c@chapter}{\newfloat{codelisting}{h}{lop}}{\newfloat{codelisting}{h}{lop}[chapter]}
\floatname{codelisting}{Listing}
\newcommand*\listoflistings{\listof{codelisting}{List of Listings}}
\makeatother
\makeatletter
\makeatother
\makeatletter
\@ifpackageloaded{caption}{}{\usepackage{caption}}
\@ifpackageloaded{subcaption}{}{\usepackage{subcaption}}
\makeatother
\usepackage{float}
\makeatletter
\let\oldlt\longtable
\let\endoldlt\endlongtable
\def\longtable{\@ifnextchar[\longtable@i \longtable@ii}
\def\longtable@i[#1]{\begin{figure}[H]
\onecolumn
\begin{minipage}{0.5\textwidth}
\oldlt[#1]
}
\def\longtable@ii{\begin{figure}[H]
\onecolumn
\begin{minipage}{0.5\textwidth}
\oldlt
}
\def\endlongtable{\endoldlt
\end{minipage}
\twocolumn
\end{figure}}
\makeatother
\journal{CPSA}
\ifLuaTeX
  \usepackage{selnolig}  % disable illegal ligatures
\fi
\usepackage[]{natbib}
\bibliographystyle{elsarticle-harv}
\usepackage{bookmark}

\IfFileExists{xurl.sty}{\usepackage{xurl}}{} % add URL line breaks if available
\urlstyle{same} % disable monospaced font for URLs
\hypersetup{
  pdftitle={Beyond Multiple Choices},
  pdfauthor={Laurence-Olivier M. Foisy; Yannick Dufresne},
  pdfkeywords={keyword1, keyword2},
  colorlinks=true,
  linkcolor={blue},
  filecolor={Maroon},
  citecolor={Blue},
  urlcolor={Blue},
  pdfcreator={LaTeX via pandoc}}

\setlength{\parindent}{6pt}
\begin{document}

\begin{frontmatter}
\title{Beyond Multiple Choices \\\large{Cleaning Open-Ended Questions
with Open Source LLMs} }
\author[1]{Laurence-Olivier M. Foisy%
\corref{cor1}%
}
 \ead{mail@mfoisy} 
\author[1]{Yannick Dufresne%
%
}
 \ead{yannick.dufresne@pol.ulaval.ca} 

\affiliation[1]{organization={Université Laval, Département de science
politique},addressline={2325 Rue de l'Université, Québec, QC G1V
0A6},city={Québec},postcode={G1V 0A6},postcodesep={}}

\cortext[cor1]{Corresponding author}


        
\begin{abstract}
Analyzing open-ended survey questions presents significant challenges
due to the diversity of responses and the manual effort required for
coding and categorization. This paper introduces a novel approach for
cleaning and analyzing open-ended questions using open-source large
language models (LLMs). Leveraging the R programming language and
Ollama's API, we demonstrate an efficient, cost-effective method for
processing qualitative data. Our approach enhances the ability to
extract meaningful insights from survey responses, providing a scalable
solution for researchers. By integrating open-source tools, we offer a
practical framework for transforming the analysis of open-ended
questions in survey research.
\end{abstract}





\begin{keyword}
    keyword1 \sep 
    keyword2
\end{keyword}
\end{frontmatter}
    
\section{Introduction}\label{introduction}

Open-ended survey questions are notoriously difficult to analyze. They
come with a host of challenges. Respondent often skip them because they
are time-consuming and require more effort and reflexion than
closed-ended questions. It can also be troublesome for mobile users to
type lengthy and complex responses \citep{dillman_etal14}. Open-ended
questions are also difficult to analyze because they require manual
coding and categorization of the responses. Indeed, many respondents can
give the same answer written in different ways. In a 2024 pilot survey
about lifestyle and health given to 2000 french and english Canadian
respondents, people were asked ``What is your favourite band or
musician?'' The most popular answer, The Beatles, was written in 10
different ways: the beatles (2), The Beatles (19), The beatles (2), The
Bwatles (1), beatles (2), Beatles (40), beetles (3), Beetles (3), les
beattels (1), Les Beatles (2). Grammatical errors, typos, and
misspellings can make it difficult to analyze the data. While it is not
impossible to analyze open-ended questions, it is generally
time-consuming and expensive, especially when dealing with large
datasets \citep{dillman_etal14, bradburn_etal04}.

However, open-ended questions can provide valuable insights into the
attitudes, opinions, and perceptions of respondents. They allow for more
detailed and nuanced responses than closed-ended questions. They avoid
the problem of forcing respondents to choose between a limited number of
options, which may not capture the full range of their opinions
\citep{dillman_etal14}. Open-ended questions can help researchers to
better understand the attitudes and opinions of respondents and to
identify emerging issues and trends.

\citet{bickman_rog09} relate open-ended questions to qualitative-data
since they require deeper analysis and interpretation, and close-ended
questions to quantitative data since they are easier to analyze and
quantify in bulk. This paper use current technology to offer an easy
method of analysis and quantification of open-ended questions with the
use of the R programming language and Ollama's API, allowing the use of
a wide array of open-source language models directly in the cleaning
process, free of charge. This method can provide valuable insights into
the data and help researchers to better understand the attitudes and
opinions of respondents.

\section{Survey Questions}\label{survey-questions}

In 1932, Rensis Likert published a seminal article in the Archives of
Psychology, introducing a new method to measure the intensity of
agreement or disagreement with a statement. The author underlined the
difficulty of measuring attitudes. He wrote that ``since it is possible
to group stimuli in almost any conceivable manner and to elassify and
subclassify them indefinitely, it is strictly true that the number of
attitudes which any given person possesses is almost infinite''
\citep{likert32}. Now known as the likert scale, this measure is widely
used in surveys and questionnaires. It allows for a standardized way to
measure attitudes, opinions, and perceptions. The likert scale is widely
used in social sciences because it is easy to administer and analyze but
it has several limitations that prevent a deeper analysis of the data.
One of the main limitations of the likert scale is that it is a
closed-ended question that does not allow for nuance or complexity in
the responses. The way survey respondents answer is completely
subjective. It forces respondents to choose between a limited number of
options, which may not capture the full range of their opinions.

\section{Methodology}\label{methodology}

\subsection{Ollama}\label{ollama}

Ollama is an open source platform that provides a user-friendly way of
downloading and running LLMs locally. It runs a server on the user's
machine that can be accessed through an API. Doing so allows the user to
interact with various LLMs without the need for extensive technical
expertise or reliance on cloud-based platforms. The Ollama API combined
with their library of pre-trained models is a powerful tool that can be
used to generate text, summarize documents, and perform a wide range of
other natural language processing tasks for free. The API is designed to
be easy to use and flexible, allowing users to customize their
interactions with LLMs to suit their needs. Using this tool, we can
easily clean and analyze open-ended survey questions with limited
resources.

Ollama offers a wide range of models in their library. Smaller models
which can be used with a typical laptop and larger models which can run
on GPU and require more computational power. Ollama recommends a minimum
of 8GB of RAM to run smaller 7 Billion parameters models and 16GB of RAM
to run larger 13 Billion parameters models \citep{ollama24}.

\subsection{The CLELLM Package}\label{the-clellm-package}

A R package that allows researchers to interact with various language
models through Ollama's API was built for this paper. A similar package
has already been published on CRAN by \citet{gruber_weber24}. However,
the one presented in this paper is a little bit more flexible, allowing
to easily change the model used between each prompts making it easier to
pick the best model for each respective task.

To install the package, users can use the following function:

\begin{Shaded}
\begin{Highlighting}[]
\NormalTok{devtools}\SpecialCharTok{::}\FunctionTok{install\_github}\NormalTok{(}\StringTok{"clessn/clellm"}\NormalTok{)}
\end{Highlighting}
\end{Shaded}

Linux users can use the following function to install ollama:

\begin{Shaded}
\begin{Highlighting}[]
\NormalTok{clellm}\SpecialCharTok{::}\FunctionTok{install\_ollama}\NormalTok{()}
\end{Highlighting}
\end{Shaded}

Windows and MacOS users can download the ollama binary from the
\href{https://ollama.com/}{Ollama website} and install it manually. Once
Ollama is installed, the user can pull the models they want to use
directly in R with the following function:

\begin{Shaded}
\begin{Highlighting}[]
\NormalTok{clellm}\SpecialCharTok{::}\FunctionTok{ollama\_install\_model}\NormalTok{(}\StringTok{"model\_name"}\NormalTok{)}
\end{Highlighting}
\end{Shaded}

The package provides a simple functions that allow users to interact
with LLMs through Ollama's API. The package is designed to be easy to
use and flexible, allowing users to interact with a wide collection of
open-source models the same way they would typically interact with
OpenAI's GPT models.

\begin{Shaded}
\begin{Highlighting}[]
\NormalTok{clellm}\SpecialCharTok{::}\FunctionTok{ollama\_prompt}\NormalTok{(}\StringTok{"prompt"}\NormalTok{, }\AttributeTok{model =} \StringTok{"model\_name"}\NormalTok{, }\AttributeTok{format =} \ConstantTok{NULL}\NormalTok{, }\AttributeTok{print\_result =} \ConstantTok{TRUE}\NormalTok{) }
\end{Highlighting}
\end{Shaded}

The functions takes in four arguments, the prompt, the model to use, the
format of the output, and whether to print the result. The prompt is the
text that the model will use to generate a response. The model is the
name of the model to use. The format is the format of the output, which
can be either ``json'' or ``text''. The print\_result argument is a
logical value that determines whether the result should be printed to
the console. The function returns the response generated by the model.

\subsection{Cleaning Open-Ended
Questions}\label{cleaning-open-ended-questions}

\section{Results}\label{results}

\section{Discussion}\label{discussion}

\section{Conclusion}\label{conclusion}


\renewcommand\refname{References}
  \bibliography{bibliography.bib}


\end{document}
